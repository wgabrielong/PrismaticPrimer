\section{Witt Vectors and Deformation Theory}\label{sec: Witt vectors}
Let $p$ be a prime number and recall the following definition. 
\begin{definition}[Perfect $\FF_{p}$-Algebra]\label{def: perfeft Fp algebra}
    Let $A$ be an $\FF_{p}$-algebra. $A$ is a perfect $\FF_{p}$ algebra if the Frobenius map $\varphi_{A}:A\to A$ by $x\mapsto x^{p}$ is an isomorphism. 
\end{definition}
The category of perfect $\FF_{p}$-algebras naturally form a category $\Perf_{\FF_{p}}$ which is a full subcategory of all $\FF_{p}$-algebras $\Alg_{\FF_{p}}$. We can construct perfect $\FF_{p}$ algebras from $\FF_{p}$ algebras by taking either a limit or colimit along constant diagrams with transition maps given by Frobenii. 
\begin{definition}[Perfection]\label{def: perfection}
    Let $A$ be an $\FF_{p}$-algebra. The perfect $\FF_{p}$-algebras $A^{\perf}$ (resp. $A_{\perf}$) are given by
    $$\lim\left(% https://q.uiver.app/#q=WzAsMyxbMCwwLCJBIl0sWzEsMCwiQSJdLFsyLDAsIlxcZG90cyJdLFswLDEsIlxcdmFycGhpX3tBfSJdLFsxLDJdXQ==
    \begin{tikzcd}
        A & A & \dots
        \arrow["{\varphi_{A}}", from=1-1, to=1-2]
        \arrow[from=1-2, to=1-3]
    \end{tikzcd}\right)\hspace{1cm}\left(\text{resp.}\colim\left(% https://q.uiver.app/#q=WzAsMyxbMCwwLCJBIl0sWzEsMCwiQSJdLFsyLDAsIlxcZG90cyJdLFswLDEsIlxcdmFycGhpX3tBfSJdLFsxLDJdXQ==
    \begin{tikzcd}
        A & A & \dots
        \arrow["{\varphi_{A}}", from=1-1, to=1-2]
        \arrow[from=1-2, to=1-3]
    \end{tikzcd}\right)\right).$$
\end{definition}
\begin{remark}
    $A\mapsto A_{\perf}$ and $A\mapsto A^{\perf}$ are the left and right adjoints of $\Perf_{\FF_{p}}\to\Alg_{\FF_{p}}$, respectively. Where unclear from the context we will refer to these as the limit and colimit perfections, respectively. 
\end{remark}
We can also construct perfect $\FF_{p}$-algebras from $p$-adically complete, $p$-torsion free $\ZZ_{p}$-algebras -- those $\ZZ_{p}$-algebras that are complete with respect to the $p$-adic norm and the kernel of the $p$th power endomorphism is trivial.
\begin{theorem}\label{thm: functor from AlgZp hat to PerfFp}
    Let $\widehat{\Alg}_{\ZZ_{p}}$ be the category of $p$-adically complete, $p$-torsion free $\ZZ_{p}$-algebras. There is a functor $\widehat{\Alg}_{\ZZ_{p}}\to\Perf_{\FF_{p}}$ by $B\mapsto B^{\flat}=(B/(p))^{\perf}$ which admits a left adjoint. 
\end{theorem}
The image of a perfect $\FF_{p}$-algebra $A$ under the left adjoint is the ring of $p$-typical Witt vectors of $A$. 
\begin{definition}[Witt Vectors]\label{def: Witt vectors}
    Let $A$ be a perfect $\FF_{p}$-algebra. The ring of $p$-typical Witt-vectors $\W(A)$ of $A$ is the $p$-adically complete, $p$-torsion free $\ZZ_{p}$-algebra that is the image of $A$ under the left adjoint to $(-)^{\flat}:\widehat{\Alg}_{\ZZ_{p}}\to\Perf_{\FF_{p}}$. 
\end{definition}
We now recall some results from deformation theory. 

Let $B$ be an $A$-algebra. We can construct the standard resolution $P_{B/A}^{\bullet}$ with terms 
$$P^{n}_{B/A}=\underbrace{A\left[A[\dots A[B]]\right]}_{n-1\text{ times}}$$
with the face and degeneracy maps given in \cite[\href{https://stacks.math.columbia.edu/tag/09CB}{Tag 09CB}]{stacks-project} which is an agumentation over $B$ as in \cite[\href{https://stacks.math.columbia.edu/tag/018G}{Tag 018G}]{stacks-project}. We define the cotangent complex as follows. 
\begin{definition}[Cotangent Complex]\label{def: cotangent complex}
    Let $A\to B$ be a morphism of rings. The cotangent complex $\LL_{B/A}$ of $A\to B$ is the complex of $B$-modules $\Omega_{P^{\bullet}_{B/A}/A}\otimes_{P^{\bullet}_{B/A},\varepsilon}B$ with augumentation $\varepsilon:P_{B/A}^{\bullet}\to B$. 
\end{definition}
\begin{remark}
    The simplicial $B$-module $\Omega_{P^{\bullet}_{B/A}/A}\otimes_{P^{\bullet}_{B/A},\varepsilon}B$ can be regarded as a chain complex by formation of the Moore complex as in \cite[\href{https://stacks.math.columbia.edu/tag/0194}{Tag 0194}]{stacks-project}.
\end{remark}
The following proposition allows us to see why the cotangent complex is an enhancement of the the classical theory of K\"{a}hler differentials. 
\begin{proposition}[{\cite[\href{https://stacks.math.columbia.edu/tag/08QF}{Tag 08QF}]{stacks-project}}]\label{prop: cotangent complex enhances Kahler differentials}
    If $A\to B$ is a morphism of rings and $\LL_{B/A}$ the cotangent complex then $H^{0}(\LL_{B/A})=\Omega_{B/A}$. 
\end{proposition}
In this light, we have the following analogue of the relative affine cotangent sequence.  
\begin{proposition}[{\cite[\href{https://stacks.math.columbia.edu/tag/08SA}{Tag 08SA}]{stacks-project}}]
    If $A\to B\to C$ are morphisms of rings with cotangent complexes $\LL_{B/A},\LL_{C/B}$ and $\LL_{C/A}$ the cotangent complex of the composite then there is a canonical distinguished triangle
    $$\LL_{B/A}\otimes^{L}_{B}C\longrightarrow \LL_{C/A}\longrightarrow\LL_{C/B}\longrightarrow\LL_{B/A}\otimes_{B}^{L}C[1]$$
    in $D(C)$. 
\end{proposition}
The main result of deformation theory is as follows. 
\begin{theorem}\label{thm: main result of deformation theory}
    Let $A$ be a ring and $\Csf_{A}$ be the category of flat $A$-algebras such that $\LL_{(-)/A}=0$.
    \begin{enumerate}[label=(\roman*)]
        \item Then for any infinitesmal thickening $B\to A$ base change to $B$ induces an equivalence of categories $\Csf_{A}\to\Csf_{B}$ hence inducing a lift of $A\hookrightarrow R$ for a flat $A$-algebra $R$ with $\LL_{R/A}=0$ to $B\hookrightarrow\widetilde{R}$. 
        \item Furthermore, for all morphisms $A\hookrightarrow R$ in $\Csf_{A}$ and surjective $A$-algebra maps $B'\to B$ with nilpotent kernel, each $A$-algebra map $R\to B$ lifts uniquely to an $A$-algebra map $A\to B'$.
    \end{enumerate}
\end{theorem}
\begin{remark}
    In particular, \Cref{thm: main result of deformation theory} (i) holds for infinitesmal thickenings of a ring $A$: surjections $B\to A$ such that $\ker(B\to A)^{n}=0$ for some $n\geq0$. 
\end{remark}
\begin{remark}
    The proof of \Cref{thm: functor from AlgZp hat to PerfFp} in fact uses \Cref{thm: main result of deformation theory} by considering infinitesmal thickenings of $\FF_{p}$. 
\end{remark}
Note that if $A\in\Perf_{\FF_{p}}$ then $\LL_{A/\FF_{p}}=0$ since the Frobenius map induces an isomorphism $\LL_{A/\FF_{p}}\to\LL_{A/\FF_{p}}$ by functoriality which is the zero map since the K\"{a}hler differentials of $A$ over $\FF_{p}$ vanish. Similarly, one can easily verify that $A$ is flat as an $\FF_{p}$-algebra (for example via \cite[\href{https://stacks.math.columbia.edu/tag/00HD}{Tag 00HD (3)}]{stacks-project}) so \Cref{thm: main result of deformation theory}applies. As such, for the infinitesmal deformation $\ZZ/p^{n}\ZZ\to\FF_{p}$ we can lift the map $\FF_{p}\hookrightarrow A$ to a map with target some $\ZZ/p^{n}\ZZ$-algebra which we define to be the $n$-truncated Witt vectors of $A$. 
\begin{definition}[$n$-Truncated Witt Vectors]\label{def: n-truncated Witt vectors}
    Let $A\in\Perf_{\FF_{p}}$. The ring of $n$-truncated Witt vectors $\W_{n}(A)$ of $A$ is the target of the unique lift of $\FF_{p}\hookrightarrow A$ in flat $\ZZ/p^{n}\ZZ$-algebras with trivial cotangent complex under the equivalence $\Csf_{\FF_{p}}\to\Csf_{\ZZ/p^{n}\ZZ}$ of \Cref{thm: main result of deformation theory} (i). 
\end{definition}
\begin{remark}
    There is an isomorphism of rings $\lim_{n}\W_{n}(A)\cong\W(A)$. 
\end{remark}
These rings of $n$-truncated Witt vectors allow us to develop a theory of Teichm\"{u}ller lifts\footnote{Oswald Teichm\"{u}ller (1913-1943) was a German mathematician who was involved in the National Socialist Party. See \href{https://mathshistory.st-andrews.ac.uk/Biographies/Teichmuller/}{\texttt{mathshistory.st-andrews.ac.uk/Biographies/Teichmuller/}} for more information.}, extending maps in $\Perf_{\FF_{p}}$ to $n$-truncated Witt vectors and thus to the ring of Witt vectors. 

Under the identification of the ring of $p$-typical Witt-vectors with sequences of elements of $A$ \cite[Def. 3.1.6, 3.2.4]{Kedlaya}, we can define the Teichm\"{u}ller lift $A\to\W(A)$ as follows. 
\begin{definition}[Teichm\"{u}ller Lift]\label{def: Teichmuller lift}
    Let $A\in\Perf_{\FF_{p}}$. The Teichm\"{u}ller lift of $A$ is the morphism $A\to\W(A)$ by $a\mapsto(a,0,\dots)$. 
\end{definition}

Let $B$ be a $p$-complete $\ZZ_{p}$ algebra and $A$ a perfect $\FF_{p}$-algebra with a map $A\to B/(p)$ and the surjection $B/(p^{n})\to B/(p)$. By \Cref{def: n-truncated Witt vectors}, we have an inclusion $\ZZ/p^{n}\ZZ\hookrightarrow\W_{n}(A)$ so applying \Cref{thm: main result of deformation theory} (ii) we have a unique $\ZZ/p^{n}\ZZ$-algebra map extending $\W_{n}(A)\to A\to B/(p)$ to $\W_{n}(A)\to B/(p^{n})$. By the universal property of the limit, this extends uniquely to a map $\W(A)\to B/(p^{n})$. 

We now consider a further specialization of the scenario above. For $B\in\widehat{\Alg}_{\ZZ_{p}}$ we have a surjection $B\to B/(p)$ and a canonical map $\overline{\theta}:B^{\flat}\to B/(p)$. By the discussion above, we can construct a map $\theta:\W(B^{\flat})\to B$. This lift in fact introduces the construction of $\A_{\inf}$ as we now define. 
\begin{definition}[$\A_{\inf}$]\label{def: Ainf}
    Let $A\in\widehat{\Alg}_{\ZZ_{p}}$. $\A_{\inf}(A)$ is defined to be $\W(A^{\flat})$
\end{definition}